\section{Related Work}\label{related-work}
I would like now to report the most relevant related work in second factor authentication. \citet{lang2016security} consulted a variety of excellent surveys work \cite{bonneau2012quest,herley2009passwords,biddle2012graphical,jain2007handbook} and listed five different technologies for 2FA, illustrating how security keys could fill some gaps in security left by those technologies. I will address four of them:
\begin{itemize}
    \item \textbf{One-Time Passcodes:} Though OTPs provide more security than passwords, OTPs have a number of downsides. First, they are vulnerable to phishing and man-in-the-middle attacks, as I cited in Sec. \ref{introduction}. Second, OTPs that are delivered by phones are subject to data and phone availability, while those that are generated by dongles cause the user to have one dongle per web site. Finally, OTPs provide a sub-optimal user experience as they often require the user to manually copy codes from one device to another. \emph{Security Keys are resistant to phishing and man-in-the-middle by design; our preliminary study also shows that they provide a better user experience.}
    \item \textbf{Smartphones as Second Factor:} While leveraging the user’s phone as a cryptographic second factor is promising, it faces a number of challenges: for example, protecting application logic from malware is difficult on a general purpose computing platform. Moreover, a user’s phone may not always be reachable: the phone may not have a data connection or the battery may have run out. \emph{Security keys require no batteries and usually have a dedicated
    tamper-proof secure element.}
    \item \textbf{TLS Client Certificates:} Unfortunately, current implementations of TLS client certificates have a poor user experience. Typically, when web servers request that browsers generate a TLS client certificate, browsers display a dialog where the user must choose the certificate cipher and key length—a cryptographic detail that is unfamiliar and confusing to most users. Accidentally choosing the wrong certificate will cause the user’s identity to leak across sites. TLS client certificates also suffer from a lack of portability: they are tough to move from one client platform to another. \emph{Security Keys have none of these issues: they are designed to be simple to use, portable and fool-proof.}
    \item \textbf{Electronic National Identification Cards:} Some countries have deployed
    national electronic identification cards. Despite their rich capabilities, national identity cards require special hardware (a card reader) and thus are hard to deploy. Moreover, they are by definition
    controlled by one government, which may not be acceptable to businesses in another country and could arise a general concern about user's privacy. \emph{Security Keys have no such downsides: they work with pre-installed drivers over commonly available physical media (USB, NFC, Bluetooth) and are not controlled or distributed by any single entity.}
\end{itemize}