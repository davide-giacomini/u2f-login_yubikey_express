\section{Introduction}\label{introduction}
In security, three main factors of authentication can be distinguished: \emph{to know}, \emph{to have} and \emph{to be}. The first one requires the user to remember something, usually a password associated to the email or a username. The second one requires the user to own something during authentication, for example their own smartphone, or a physical key. The latter implies to recognize the user based on something that characterizes their uniqueness, for example from a fingerprint or their face.

Although passwords are inherently the weakest form of security to consider, they are actually the most widespread as form of user authentication on the web today \cite{reynolds2018tale}. Several are the techniques used to violate a password-protected account, among them there is phishing, guessing or spoofing \cite{reynolds2018tale}. Therefore, many have advocated for the use of a second factor for authenticating (2FA: 2 Factor Authentication) \cite{bonneau2012quest}, thus limiting guessing and spoofing. Biometric authentication is becoming more and more widespread, especially with fingerprint scanners on phone, and in the last years with face scanner too. There are although several issues still not widely studied in biometric authentication, and systems still have defects that bring to their vulnerability. \citet{bonneau2012quest} surveyed these issues and also considered that some biological features have not been deeply studied yet.

Despite 2FA is extremely effective for better secure accounts, there are still attacks that can be carried out. For instance, OTPs can be victim of the so called real-time phishing attacks, which require the attacker to act as a ``server-in-the-middle'', pretending to be the real server and forwarding user's authentication requests and user's OTP to the real server \cite{reailton2015phishing}. However, message-based OTPs are the most used 2FA nowadays, especially because they rely on the assumption that a user will always carry their smartphone with them.

In this report, I will explain why security keys are theoretically better rather than other 2FA methods common today, bringing up problems in using other second factors and showing how they are avoided with a security key. I will then go through a sample web application that I developed in NodeJS\footnote{\url{https://nodejs.org/en/}} and VueJS\footnote{\url{https://vuejs.org/}}, using Express\footnote{\url{https://expressjs.com/}} as framework, after looked at an overview of the cryptographic protocol used by Yubico\footnote{\url{https://yubico.com}} for their security keys.